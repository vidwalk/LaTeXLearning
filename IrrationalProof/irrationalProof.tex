\documentclass[11pt, a4paper]{article}
\title{{$\sqrt{3}$ is Irrational Proof}}
\date{15-08-2019}
\author{Claudiu Rediu}
\usepackage[margin=1.5in]{geometry}
\usepackage{hyperref}
\usepackage[utf8]{inputenc}
\usepackage[english]{babel}
\usepackage{amsthm}
\usepackage{amssymb}
\usepackage{amsmath}
\usepackage{graphicx}
\renewcommand\qedsymbol{$\blacksquare$}
\usepackage{float}
\theoremstyle{definition}
\newtheorem*{definition}{Definition} %the * is such that the definition are not numbered
\theoremstyle{theorem}
\newtheorem*{theorem}{Proposition}

\begin{document}
	\pagenumbering{gobble}
	\maketitle
	\newpage
	\pagenumbering{arabic}
	\newpage
	\begin{theorem}
		$\sqrt{3}$ is irrational.
	\end{theorem}

	\begin{proof}
	

		Suppose $\sqrt{3}$ is rational, such that $$\sqrt{3} = \frac{a}{b} \ \ \forall a,b \in\mathbb{Z}.$$
		If we square the fraction, we get $3 = \frac{a^2}{b^2}$.
		Without loss of generality, we can suppose that $a^2$ is even and $b^2$ odd. We do this because 3 would result from a fraction with integers of the same type. We set $a^2 = 2k$ and $b^2 = 2j+1 \ \forall k,j \in\mathbb{Z}$.
		\begin{equation}
		\begin{split}
		3 &= \frac{2k}{2j+1} \\
		6j+3 &= 2k \\
		3j + \frac{3}{2} &= k \implies k \notin\mathbb{Z}. 
		\end{split}
		\end{equation}
		Then $3 = \frac{a^2}{b^2}$ and $3 \ne \frac{a^2}{b^2}$ for $\forall a,b \in\mathbb{Z}$. From this we can conclude that $\sqrt{3}$ is irrational.
	\end{proof}

\end{document}
\documentclass{article}

\title{\Huge Food Origins for Central and South Europeans}
\date{01-04-2020}
\author{Claudiu Rediu}
\usepackage{fancyhdr} % Customizable Headers
\usepackage{hyperref}
\usepackage[toc,page]{appendix}
\pagestyle{fancy}
\usepackage{graphicx}
\fancyhf{}
\lhead{version 1.0}
\rfoot{Page \thepage}
\begin{document}
	\pagenumbering{gobble}
	\maketitle
	\newpage
	\pagenumbering{arabic}
	\newpage
	\section{Introduction}
	The topic we are going to look into is food origins and why they are relevant. Adaptations are influenced by the environment and random occurrences. Adaptability for a specific food evolves over time. Breaking down food requires that our body knows what to do with it. From this we can form the assumption that the best food for us is the one we are most used to. \newline
	Agricultural communities appeared around 12000 years ago. This change of behaviour influence human's adaptation to what they can consume. It is safe to say that what we started to eat then, we still eat now and probably will eat in the future. Tolerance for those foods has been built as system to digest them as well as possible.\newline
	This brings us to the main topic, which is food origins for Central and South Europeans. Why specifically this group? Because genes have move around in very few quantities. What spread was languages, but not genetic material.  
	\section{Foods}
	The foods that Central and South Europeans have best tolerance is those they came into contact most. Those are:
	\begin{itemize}
		\item Europe
		\begin{itemize}
			\item Fruit
			\begin{itemize}
				\item Blackcurrant
				\item Damsons(Plum)
				\item Juniper berry
				\item Pear
				\item Raspberry
				\item Bilberry
			\end{itemize}
		\end{itemize}
		\begin{itemize}
			\item Vegetables
			\begin{itemize}
				\item Angelica
				\item Cabbage
				\item Parsnips
				\item Radish
				\item Rapeseed
				\item Turnip
			\end{itemize}
		\end{itemize}
		\begin{itemize}
			\item Herbs
			\begin{itemize}
				\item Caraway
				\item Dill
				\item Hops
				\item Tarragon
				\item Thyme
				\item Oregano
				\item Wormwood
			\end{itemize}
		\end{itemize}
		\begin{itemize}
			\item Others
			\begin{itemize}
				\item Chestnuts
			\end{itemize}
		\end{itemize}
		\begin{itemize}
			\item Meat
			\begin{itemize}
				\item Duck
				\item Rabbit
				\item Pork
				\item Beef
			\end{itemize}
		\end{itemize}
	\item Mediterranean
		\begin{itemize}
			\item Vegetables, Cereals, etc.
			\begin{itemize}
				\item Wheat
				\item Barley
				\item Millet
				\item Pea
				\item Broad Bean
				\item Lentil
				\item Flax
				\item Sesame
				\item Chickpea
				\item Hemp
				\item Turnip
				\item Beets
				\item Broccoli
				\item Brussels sprouts
				\item Caper
				\item Catnip
				\item Cauliflower
				\item Centaurium
				\item Fennel
				\item Kale
				\item Kohlrabi
			\end{itemize}
		\end{itemize}
		\begin{itemize}
			\item Fruits
			\begin{itemize}
				\item Black mulberry
				\item Cornelian cherry 
				\item Date palm 
				\item Fig
				\item Grape
				\item Jujube
				\item Olive
				\item Pomegranate
				\item Sycamore Fig
			\end{itemize}
		\end{itemize}
	\end{itemize}
\end{document}